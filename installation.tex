\documentclass[12pt]{article}
\usepackage{enumerate, amsmath, amsthm, amssymb,multirow}
\usepackage{graphicx}

\addtolength{\oddsidemargin}{-.5in}
\addtolength{\evensidemargin}{-.5in} \addtolength{\textwidth}{1.0in}
\begin{document}

\title{Installation Instructions}

\maketitle

\section{Install Julia}

Download julia-0.4.2 from the website \texttt{julialang.org}.  Run the executable.  

Open up a julia terminal.  Install several packages using the following code:\\

Pkg.add(``JuMP'')

Pkg.add(``DataFrames'')

Pkg.add(``Gadfly'')

Pkg.add(``Cairo'')\\

Make sure the following code works:\\

using JuMP

using DataFrames

using Gadfly

using Cairo

\section{Install Sublime Text 3}

Download Sublime Text 3 from the website \texttt{sublimetext.org}.  Run the executable.

Install the sublime-julia package.  

\section{Set your Julia working directory}

Create a new folder on your desktop where you can save all of your julia programs.  Name it something simple like ``optimization''.  Open up a new file in Sublime Text 3.  Add the path to the working directory to this file, using the command: cd(``path\_here'').  For example, it might be: cd("C:/Users/colin/Documents/optimization").  Windows users, use ``//'' instead of ``/'' in the path name.  \\

Save this file as ``.juliarc.jl'' in the same folder as your ``.julia\_history.jl'' file.  

\section{Command line tips}
In the Julia terminal, use {\bf ;} to switch to shell commands.  This allows you to navigate to different directories in the computer.  {\bf ls} lists the files in the current directory.  \\

{\bf cd} stands for ``change directory''.  {\bf cd} \emph{folder} opens up \emph{folder} in the present directory, and {\bf cd ..} goes up one level. \\

Test out these commands.  Open up the Julia terminal, and type in {\bf ;}, then {\bf ls}.  The files in your current folder, ``optimization'', should pop up.  Pop out of the folder by typing in {\bf ;}, then {\bf cd ..}.  Type in {\bf ;}, then {\bf ls} to see
all of the folders in that directory.  Return to the directory by typing in {\bf ;}, then {\bf cd optimization}.\\

To run a julia file, use the command {\bf include(``filename.jl'')}.  You can use this command as long as ``filename.jl'' is in the current
directory.  
\end{document}