\documentclass[12pt]{article}
\usepackage{enumerate, amsmath, amsthm, amssymb,multirow}
\usepackage{graphicx}
\usepackage{hyperref}


\addtolength{\oddsidemargin}{-.5in}
\addtolength{\evensidemargin}{-.5in} \addtolength{\textwidth}{1.0in}
\begin{document}

\title{Installation Instructions}

\maketitle

\section{Install Julia}

\begin{enumerate}

\item Download julia-0.4.2 from \url{http://julialang.org/downloads/}. 

\item Open up a julia terminal.  

\item Install several packages using the following code, line by line:

Pkg.add(``JuMP'') \\
Pkg.add(``DataFrames'')\\
Pkg.add(``Gadfly'')\\
Pkg.add(``Cairo'')
%\\ Pkg.add(``ZMQ'')\\
%Pkg.add(``IJulia'')\\

\item Make sure the following code works:

using JuMP \\
using DataFrames \\
using Gadfly \\
using Cairo 
%\\ using ZMQ \\
%using IJulia \\

\item At this point, you have the Julia programming language successfully installed on your computer!  In the terminal prompt, you can experiment by typing in some math and then ``enter''.  For example, you can try: ``2+2'', ``3\string^2'', and ``12*10''.  
\end{enumerate}

\section{Install Sublime Text 2}

\begin{enumerate}

\item Download Sublime Text 2 from \url{http://www.sublimetext.com/2}.  

\item Go to the webpage \url{https://packagecontrol.io/installation\#Manual} and copy the code block under ``Sublime Text 2''.

\item Open up Sublime Text 2 and click on ``View $\rightarrow$ Show Console''.

\item Paste the code block and hit ``enter''.

\item Restart Sublime Text 2.  

\item Press ``ctrl + shift + p'' (Windows) or ``cmd + shift + p'' (Mac) to open up a search bar.  Begin typing ``Install Package'' and click on ``Package Control: Install Package''.  

\item Begin typing ``Julia'' and click on ``Julia''.  The package for Julia syntax highlighting will automatically install.  

\item Open up ``jump\_start.jl''.  Click on ``View $\rightarrow$ Syntax $\rightarrow$ Julia'' to view the code with syntax highlighting.  

%\item (Optional) Click on ``Preferences $\rightarrow$ Color Scheme $\rightarrow$ Eiffel'' to change the color scheme.  

\end{enumerate}

\section{Set your Julia working directory}

\begin{enumerate}

\item Create a new folder on your desktop where you can save all of your Julia programs.  Name it something simple like ``optimization''.  

\item Create a new file in Sublime Text 2 named ``.juliarc.jl''.  

\item In this file, type: cd(``\emph{path\_here}'').  In place of \emph{path\_here}, put the file path on the computer to the ``optimization'' folder.  For example, it might be: \\\\
cd (``C:$\backslash\backslash$Users$\backslash\backslash$colin$\backslash\backslash$Documents$\backslash\backslash$optimization").  

\item Open up a new Julia terminal and type ``homedir()'' to determine the home directory folder.   

\item Save ``.juliarc.jl'' in the home directory folder, which is the output of the ``homedir()'' command run in the previous step.   

\item To test that the working directory is set up correctly, open up a new Julia terminal and type in {\bf ;} and then {\bf pwd}.  Check that the folders match up.  

\end{enumerate}

\section{Command line tips}
In the Julia terminal, use {\bf ;} to switch to shell commands, and {\bf backspace} to switch back\footnote{These are also called ``command line'' or ``BaSH'' commands. }.  These commands allow you to navigate to different directories in the computer.   Here are a few of the most useful ones:\\

{\bf ls} lists the files in the current directory.  \\

{\bf cd} stands for ``change directory''.  {\bf cd} \emph{folder} opens up \emph{folder} in the present directory, and {\bf cd ..} goes up one level. \\

{\bf pwd} stands for ``pass working directory''.  This tells you which directory you are currently in.  \\

Test out these commands below.  

\begin{enumerate}

\item (Optional) Open up the Julia terminal, and type in {\bf ;}, then {\bf ls}.  The files in your current folder, ``optimization'', should pop up.  

\item (Optional) Pop out of the folder by typing in {\bf ;}, then {\bf cd ..}.  

\item (Optional) Type in {\bf ;}, then {\bf ls} to see all of the folders in that directory.  

\item (Optional) Return to the directory by typing in {\bf ;}, then {\bf cd optimization}.

\end{enumerate}

To run a julia file, use the command {\bf include(``filename.jl'')}.  You can use this command as long as ``filename.jl'' is in the current
directory.  

\section{Julia tips}

\begin{enumerate}

\item (Optional) Visit the webpage \url{http://docs.julialang.org/en/release-0.4/} for an introduction to the Julia programming language.  

\item (Optional) For a condensed list of many useful Julia commands, see ``Julia-cheatsheet.pdf''.  

\item (Optional) When you are typing in code in the Julia terminal, you can use ``tab'' to auto-complete certain phrases.  This is especially useful for shell commands.  

\item {\bf Experiment and have fun!}  You learn programming languages by trying things out yourself, looking stuff up on the internet, and writing your own code.  
\end{enumerate}
\end{document}